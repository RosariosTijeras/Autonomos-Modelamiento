\documentclass[12pt, letterpaper]{article}

% ============================================
% PAQUETES OPTIMIZADOS - APA v7 COMPLETO
% ============================================
\usepackage[utf8]{inputenc}
\usepackage[spanish]{babel}

% Bibliografía APA v7 (optimizada)
\usepackage{csquotes}
\usepackage[style=apa, backend=biber, sorting=nyt]{biblatex}
\addbibresource{bibliografia.bib}

% Paquetes matemáticos (esenciales)
\usepackage{amsmath}
\usepackage{amsfonts}
\usepackage{amssymb}

% Paquetes de gráficos y tablas (optimizados)
\usepackage{graphicx}
\usepackage{float}
\usepackage{booktabs}
\usepackage{array}
\usepackage{multirow}
\usepackage{caption}
\usepackage{subcaption}

% Paquetes para manejo de texto y márgenes
\usepackage{microtype}      % Mejora el espaciado y evita desbordamientos
\usepackage{ragged2e}       % Mejor justificación de texto
\usepackage{hyphenat}       % Control de separación de palabras
\usepackage{url}            % Para URLs que pueden ser muy largas
\usepackage{xcolor}         % Para colores (necesario para algunos paquetes)
\usepackage{enumitem}       % Control avanzado de listas
\usepackage{calc}           % Para cálculos en longitudes
\usepackage{array}          % Para manejo avanzado de tablas y alineación

% ============================================
% CONFIGURACIÓN APA v7: MÁRGENES Y ESPACIADO
% ============================================
\usepackage{geometry}
\geometry{
    letterpaper,
    top=2.54cm,    % 1 pulgada (APA v7)
    bottom=2.54cm, % 1 pulgada (APA v7)
    left=2.54cm,   % 1 pulgada (APA v7)
    right=2.54cm   % 1 pulgada (APA v7)
}

% Doble espaciado (APA v7)
\usepackage{setspace}
\doublespacing

% ============================================
% CONFIGURACIÓN PARA EVITAR DESBORDAMIENTOS
% ============================================
% Configuración para evitar overfull hbox con texto largo
\tolerance=2000
\emergencystretch=5em
\hbadness=3000
\linepenalty=4
\pretolerance=10000

% Configuración de microtype para mejor tipografía
\microtypesetup{
    expansion=true,
    protrusion=true,
    tracking=true,
    kerning=true,
    spacing=true
}

% Configuración adicional para manejo de texto largo
\setlength{\rightskip}{0pt plus 2em}
\setlength{\parfillskip}{0pt plus 1fil}

% Permitir separación de palabras con guión en palabras compuestas
% NOTA: \hyphenation no funciona con palabras acentuadas en español
% Se usa babel para manejar separación automática de palabras en español
\hyphenation{
    Universidad
    Nacional
    Chimborazo
    Ingenieria
    Modelamiento
    estadistico
    analisis
    distribucion
    correlacion
}

% ============================================
% CONFIGURACIÓN DE LISTAS PARA EVITAR DESBORDAMIENTOS
% ============================================
% Configuración global para todas las listas con mejor manejo de texto largo
\setlist[enumerate]{
    leftmargin=*,
    itemsep=0.3\baselineskip,
    parsep=0pt,
    topsep=0.3\baselineskip,
    labelsep=0.5em,
    listparindent=\parindent,
    itemindent=0pt,
    align=left,
    widest=9
}

\setlist[itemize]{
    leftmargin=*,
    itemsep=0.3\baselineskip,
    parsep=0pt,
    topsep=0.3\baselineskip,
    labelsep=0.5em,
    listparindent=\parindent,
    itemindent=0pt
}

% Configuración específica para listas anidadas
\setlist[enumerate,2]{
    leftmargin=*,
    itemsep=0.2\baselineskip,
    label=\alph*)
}

\setlist[itemize,2]{
    leftmargin=*,
    itemsep=0.2\baselineskip,
    label=--
}

% ============================================
% CONFIGURACIÓN APA v7: ENCABEZADOS Y PIE DE PÁGINA
% ============================================
\usepackage{fancyhdr}
\setlength{\headheight}{15pt} % Corregir altura del encabezado
\pagestyle{fancy}
\fancyhf{} % Limpiar encabezados

% Running head (encabezado corto)
\fancyhead[L]{} % Se configurará después para páginas normales
\fancyhead[R]{\thepage} % Número de página arriba a la derecha (APA v7)

% Línea del encabezado
\renewcommand{\headrulewidth}{0pt} % Sin línea (APA v7)

% ============================================
% CONFIGURACIÓN APA v7: TÍTULOS Y SECCIONES
% ============================================
\usepackage{titlesec}

% Nivel 1: Centrado, Negrita, Capitalizado
\titleformat{\section}
{\normalfont\large\bfseries\centering}{\thesection}{1em}{}

% Nivel 2: Alineado a la izquierda, Negrita
\titleformat{\subsection}
{\normalfont\normalsize\bfseries\raggedright}{\thesubsection}{1em}{}

% Nivel 3: Alineado a la izquierda, Negrita, Cursiva
\titleformat{\subsubsection}
{\normalfont\normalsize\bfseries\itshape\raggedright}{\thesubsubsection}{1em}{}

% ============================================
% CONFIGURACIÓN APA v7: SANGRÍA DE PÁRRAFO
% ============================================
\setlength{\parindent}{1.27cm} % 0.5 pulgadas (APA v7)

% ============================================
% INFORMACIÓN DEL DOCUMENTO (para portada APA)
% ============================================
\newcommand{\runtitulo}{[Título Corto del Trabajo]}
\newcommand{\titulo}{[Título Completo del Trabajo]}
\newcommand{\autores}{[Nombre Autor 1, Nombre Autor 2, Nombre Autor 3]}
\newcommand{\afiliacion}{[Nombre de la Universidad]}
\newcommand{\facultad}{[Nombre de la Facultad]}
\newcommand{\carrera}{[Nombre de la Carrera]}
\newcommand{\materia}{[Nombre de la Materia]}
\newcommand{\docente}{[Nombre del Docente]}
\newcommand{\semestre}{[Número del Semestre]}
\newcommand{\ciudad}{[Ciudad - País]}

\begin{document}
% ============================================
% PORTADA APA v7
% ============================================
\begin{titlepage}
    \thispagestyle{fancy}
    \fancyhf{}
    \fancyhead[R]{\thepage}

    \centering

    \vspace{0.5cm}
    \begin{figure}[H]
        \centering
        \includegraphics[width=1\linewidth]{../Graficos/Imagen1.png}
    \end{figure}

    % Título (doble espaciado, negrita, centrado)
    {\Large\bfseries \titulo \par}

    \vspace{2cm}

    % Nombre de los autores
    {\large \autores \par}

    \vspace{0.5cm}

    % Afiliación institucional
    {\large \afiliacion \par}
    {\normalsize \facultad \par}
    {\normalsize \carrera \par}

    \vspace{1cm}

    % Información del curso
    {\normalsize \textbf{Materia:} \materia \par}
    {\normalsize \textbf{Docente:} \docente \par}
    {\normalsize \textbf{Semestre:} \semestre \par}

    \vfill

    % Fecha y lugar
    {\normalsize \ciudad \par}
    {\normalsize \today \par}
\end{titlepage}

% ============================================
% CUERPO DEL DOCUMENTO
% ============================================
\newpage
\setcounter{page}{2}

\section{Introducción}

 [Escribir la introducción del trabajo. Describir el contexto del problema, su relevancia y justificación. Incluir antecedentes del tema y la importancia del estudio en el área específica.]

\subsection{Planteamiento del Problema}

[Describir claramente el problema que se abordará en este trabajo. Explicar la situación problemática, sus causas y consecuencias.]

\subsection{Preguntas de Investigación}

\begin{enumerate}
    \item [Escribir la primera pregunta de investigación específica]
    \item [Escribir la segunda pregunta de investigación específica]
    \item [Agregar más preguntas si es necesario]
\end{enumerate}

\subsection{Objetivos}

\subsubsection{Objetivo General}

[Escribir el objetivo general del trabajo - debe ser amplio y abarcar el propósito principal del estudio]

\subsubsection{Objetivos Específicos}

\begin{enumerate}
    \item [Escribir el primer objetivo específico - debe ser medible y alcanzable]
    \item [Escribir el segundo objetivo específico - debe contribuir al objetivo general]
    \item [Escribir el tercer objetivo específico - puede incluir aspectos metodológicos]
    \item [Agregar más objetivos específicos según sea necesario]
\end{enumerate}

\subsection{Hipótesis}

\textbf{Hipótesis nula ($H_{0}$):} [Escribir la hipótesis nula]

\textbf{Hipótesis alternativa ($H_{1}$):} [Escribir la hipótesis alternativa]

Donde [explicar los parámetros]. Para la prueba de hipótesis se empleará
un nivel de significancia de $\alpha = 0.05$.

\section{Marco Teórico}

 [Desarrollar el marco teórico con referencias bibliográficas relevantes. Incluir conceptos fundamentales, teorías relacionadas y estudios previos. Para citar usar: \textcite{autor2023} para narrativas o \parencite{autor2023} para parentéticas]

\subsection{[Primer Concepto o Teoría Importante]}

[Desarrollar el primer concepto teórico fundamental para el estudio]

\subsection{[Segundo Concepto o Teoría Importante]}

[Desarrollar el segundo concepto teórico relevante]

\subsection{[Estudios Previos o Antecedentes]}

[Revisar literatura previa y estudios relacionados con el tema]

\section{Metodología}

\subsection{Datos}

[Describir el dataset utilizado, sus características y procedencia. Incluir información sobre el tamaño de la muestra, variables incluidas, criterios de selección y limitaciones de los datos]

\subsection{Métodos Estadísticos}

[Describir los métodos estadísticos que se utilizarán. Explicar la justificación para la selección de cada método y cómo se relacionan con los objetivos del estudio]

\subsection{Criterios de Inclusión y Exclusión}

[Si aplica, describir los criterios utilizados para incluir o excluir observaciones del análisis]

\subsection{Software y Herramientas}

[Describir el software y las herramientas utilizadas para el análisis. Ejemplo:]

El análisis estadístico se realizó utilizando [nombre del software], empleando los siguientes paquetes/librerías especializadas:

\begin{itemize}
    \item [Nombre del paquete/librería 1]: [Descripción de su uso en el análisis]
    \item [Nombre del paquete/librería 2]: [Descripción de su uso en el análisis]
    \item [Nombre del paquete/librería 3]: [Descripción de su uso en el análisis]
    \item [Agregar más según sea necesario]
\end{itemize}

\section{Resultados}

 [Introducir la sección de resultados explicando la estructura del análisis y los principales hallazgos que se presentarán]

\subsection{Exploración del Dataset}

[Describir las características generales del dataset analizado]

El análisis se realizó utilizando el dataset [nombre del dataset], que contiene información relevante para los objetivos planteados. El dataset seleccionado presenta las siguientes características:

\begin{itemize}
    \item \textbf{[Nombre de Variable 1]}: [Descripción de la variable y su rol en el estudio]
    \item \textbf{[Nombre de Variable 2]}: [Descripción de la variable y su rol en el estudio]
    \item \textbf{[Agregar más variables según sea necesario]}
\end{itemize}

[Describir los resultados de la exploración inicial, incluyendo completitud de datos, valores faltantes, etc.]

\begin{table}[H]
    \centering
    \caption{Estructura y Completitud del Dataset}
    \begin{tabular}{@{}l>{\centering}p{2.5cm}>{\centering}p{2.5cm}>{\centering\arraybackslash}p{2cm}@{}}
        \toprule
        \textbf{Variable} & \textbf{Observa-} \textbf{ciones} & \textbf{Valores} \textbf{Perdidos} & \textbf{Comple-} \textbf{titud} \\
        \midrule
        {[}Variable 1{]}  & {[}n{]}                           & {[}número{]}                       & {[}porcentaje\%{]}              \\
        {[}Variable 2{]}  & {[}n{]}                           & {[}número{]}                       & {[}porcentaje\%{]}              \\
        {[}Variable 3{]}  & {[}n{]}                           & {[}número{]}                       & {[}porcentaje\%{]}              \\
        \midrule
        \textbf{Total}    & {[}n{]}                           & {[}número total{]}                 & {[}porcentaje\%{]}              \\
        \bottomrule
    \end{tabular}
    \label{tab:estructura}

    \vspace{0.2cm}
    \textit{Nota.} [Agregar nota explicativa sobre la tabla y los datos presentados]
\end{table}

\subsection{Análisis Descriptivo}

[Describir el propósito del análisis descriptivo y los estadísticos que se calcularán]

Se calcularon medidas de tendencia central y dispersión para caracterizar las variables de estudio. La Tabla \ref{tab:descriptivas} resume estos estadísticos descriptivos.

\begin{table}[H]
    \centering
    \caption{Estadísticas Descriptivas de las Variables de Estudio}
    \resizebox{\textwidth}{!}{%
        \begin{tabular}{@{}lcccccc@{}}
            \toprule
            \textbf{Variable} & \textbf{Media} & \textbf{Mediana} & \textbf{DE} & \textbf{Mín} & \textbf{Máx} & \textbf{n} \\
            \midrule
            {[}Variable 1{]}  & {[}valor{]}    & {[}valor{]}      & {[}valor{]} & {[}valor{]}  & {[}valor{]}  & {[}n{]}    \\
            {[}Variable 2{]}  & {[}valor{]}    & {[}valor{]}      & {[}valor{]} & {[}valor{]}  & {[}valor{]}  & {[}n{]}    \\
            {[}Variable 3{]}  & {[}valor{]}    & {[}valor{]}      & {[}valor{]} & {[}valor{]}  & {[}valor{]}  & {[}n{]}    \\
            \bottomrule
        \end{tabular}%
    }
    \label{tab:descriptivas}

    \vspace{0.2cm}
    \textit{Nota.} DE = Desviación Estándar; Mín = Mínimo; Máx = Máximo; n = Tamaño de muestra.
\end{table}

\begin{figure}[H]
    \centering
    % \includegraphics[width=0.45\textwidth]{../Graficos/[nombre_grafico1].png}
    % \hfill
    % \includegraphics[width=0.45\textwidth]{../Graficos/[nombre_grafico2].png}
    \rule{5cm}{3cm} % Placeholder para gráfico
    \hfill
    \rule{5cm}{3cm} % Placeholder para gráfico
    \caption{[Título descriptivo de los gráficos mostrados]}
    \label{fig:[etiqueta_descriptiva]}
\end{figure}

Los hallazgos principales del análisis descriptivo incluyen:

\begin{itemize}
    \item [Descripción del primer hallazgo principal del análisis descriptivo]
    \item [Descripción del segundo hallazgo importante observado en los datos]
    \item [Descripción del tercer hallazgo relevante o patrón identificado]
    \item [Agregar más hallazgos según sea necesario]
\end{itemize}

\subsection{[Nombre de la Prueba o Análisis Específico]}

[Describir el propósito de esta sección de análisis específica]

Se evaluó [descripción del análisis] mediante [nombre de la prueba/método], cuya hipótesis nula establece que [descripción de H0].

\begin{table}[H]
    \centering
    \caption{Resultados de [Nombre de la Prueba]}
    \begin{tabular}{@{}l>{\centering}p{2.5cm}>{\centering}p{2cm}>{\centering\arraybackslash}p{3cm}@{}}
        \toprule
        \textbf{Variable} & \textbf{Estadístico} \textbf{{[}Símbolo{]}} & \textbf{p-valor} & \textbf{Decisión} \textbf{($\alpha = 0.05$)} \\
        \midrule
        {[}Variable 1{]}  & {[}valor{]}                                 & {[}valor{]}      & {[}Decisión{]}                               \\
        {[}Variable 2{]}  & {[}valor{]}                                 & {[}valor{]}      & {[}Decisión{]}                               \\
        {[}Variable 3{]}  & {[}valor{]}                                 & {[}valor{]}      & {[}Decisión{]}                               \\
        \bottomrule
    \end{tabular}
    \label{tab:[etiqueta_tabla]}

    \vspace{0.2cm}
    \textit{Nota.} [Agregar nota explicativa sobre la interpretación de los resultados]
\end{table}

\subsection{[Nombre del Análisis de Correlación/Asociación]}

[Describir el propósito del análisis de correlación o asociación]

Se calculó [tipo de coeficiente] para evaluar la asociación [lineal/no lineal] entre las variables de interés:

\begin{table}[H]
    \centering
    \caption{Resultados del Análisis de [Tipo de Correlación]}
    \begin{tabular}{@{}lc@{}}
        \toprule
        \textbf{Estadístico}                              & \textbf{Valor} \\
        \midrule
        Coeficiente de {[}tipo{]} ($\{[\}$símbolo$\{]\}$) & {[}valor{]}    \\
        Estadístico $[letra]$                             & [valor]        \\
        Grados de libertad                                & [valor]        \\
        p-valor                                           & [valor]        \\
        Intervalo de confianza [porcentaje]\%             & [intervalo]    \\
        \bottomrule
    \end{tabular}
    \label{tab:[etiqueta_correlacion]}
\end{table}

\begin{figure}[H]
    \centering
    % \includegraphics[width=0.75\textwidth]{../Graficos/[nombre_grafico_correlacion].png}
    \rule{8cm}{5cm} % Placeholder para gráfico de correlación
    \caption{[Título descriptivo del gráfico de correlación]}
    \label{fig:[etiqueta_correlacion]}
\end{figure}

\subsection{[Nombre del Modelo Estadístico]}

Se calculó el coeficiente de correlación de Pearson para evaluar la asociación lineal
entre las variables de interés:

\begin{table}[H]
    \centering
    \caption{Resultados del Análisis de Correlación de Pearson}
    \begin{tabular}{@{}lc@{}}
        \toprule
        \textbf{Estadístico}             & \textbf{Valor} \\
        \midrule
        Coeficiente de correlación ($r$) & [valor]        \\
        Estadístico $t$                  & [valor]        \\
        Grados de libertad               & [valor]        \\
        p-valor                          & [valor]        \\
        Intervalo de confianza 95\%      & [intervalo]    \\
        \bottomrule
    \end{tabular}
    \label{tab:correlacion}
\end{table}

\begin{figure}[H]
    \centering
    % \includegraphics[width=0.75\textwidth]{../Graficos/correlacion.png}
    \rule{8cm}{5cm} % Placeholder para gráfico de correlación
    \caption{Gráfico de dispersión con línea de tendencia}
    \label{fig:correlacion}
\end{figure}

\subsection{Modelo Estadístico}

Se ajustó un modelo estadístico apropiado para responder a las preguntas de investigación.
El modelo se puede expresar como:

\begin{equation}
    Y = \beta_0 + \beta_1 X + \varepsilon
    \label{eq:modelo}
\end{equation}

Donde $Y$ es la variable dependiente, $X$ es la variable independiente, $\beta_0$ es el
intercepto, $\beta_1$ es la pendiente y $\varepsilon$ es el término de error.

\subsubsection{Estimación de Coeficientes}

\begin{table}[H]
    \centering
    \caption{Coeficientes del Modelo Estadístico}
    \begin{tabular}{@{}lcccc@{}}
        \toprule
        \textbf{Coeficiente}   & \textbf{Estimación} & \textbf{EE} & \textbf{Estadístico t} & \textbf{p-valor} \\
        \midrule
        $\beta_0$ (Intercepto) & [valor]             & [valor]     & [valor]                & [valor]          \\
        $\beta_1$ (Pendiente)  & [valor]             & [valor]     & [valor]                & [valor]          \\
        \bottomrule
    \end{tabular}
    \label{tab:coeficientes}

    \vspace{0.2cm}
    \textit{Nota.} EE = Error Estándar.
\end{table}

\subsubsection{Bondad de Ajuste}

\begin{table}[H]
    \centering
    \caption{Métricas de Bondad de Ajuste del Modelo}
    \begin{tabular}{@{}lc@{}}
        \toprule
        \textbf{Métrica}                     & \textbf{Valor} \\
        \midrule
        Coeficiente de determinación ($R^2$) & [valor]        \\
        Error estándar residual              & [valor]        \\
        Estadístico F                        & [valor]        \\
        p-valor (prueba F)                   & [valor]        \\
        \bottomrule
    \end{tabular}
    \label{tab:bondad}
\end{table}

\subsubsection{Diagnóstico de Residuos}

\begin{figure}[H]
    \centering
    % \includegraphics[width=0.8\textwidth]{../Graficos/residuos.png}
    \rule{10cm}{6cm} % Placeholder para gráficos de diagnóstico
    \caption{Gráficos de diagnóstico del modelo}
    \label{fig:residuos}
\end{figure}

\section{Discusión}

\subsection{Interpretación de los Hallazgos}

[Introducir la sección de discusión explicando cómo se interpretarán los resultados]

Los resultados del análisis proporcionan evidencia para evaluar las hipótesis planteadas. Los principales hallazgos se interpretan de la siguiente manera:

\begin{enumerate}
    \item [Interpretación del primer hallazgo principal en relación con la literatura existente]
    \item [Interpretación del segundo hallazgo principal y sus implicaciones teóricas]
    \item [Interpretación de las implicaciones estadísticas y su significado práctico]
    \item [Agregar más interpretaciones según sea necesario]
\end{enumerate}

\subsection{Limitaciones del Estudio}

Es importante reconocer las siguientes limitaciones del presente estudio:

\begin{itemize}
    \item [Primera limitación metodológica identificada y su impacto en los resultados]
    \item [Segunda limitación relacionada con los datos o la muestra utilizada]
    \item [Tercera limitación del diseño del estudio o las técnicas analíticas empleadas]
    \item [Agregar más limitaciones según sea necesario]
\end{itemize}

\subsection{Implicaciones}

Los resultados tienen implicaciones importantes tanto desde el punto de vista teórico como práctico:

\begin{itemize}
    \item [Implicación teórica principal y su contribución al conocimiento del área]
    \item [Implicación práctica principal y su aplicabilidad en contextos reales]
    \item [Implicaciones para futuras investigaciones y direcciones de estudio]
    \item [Agregar más implicaciones según sea necesario]
\end{itemize}

\section{Conclusiones}

\subsection{Respuestas a las Preguntas de Investigación}

Con base en los resultados obtenidos, se responden las preguntas planteadas inicialmente:

\textbf{Pregunta 1:} [Repetir la primera pregunta de investigación]

\textbf{Respuesta:} [Proporcionar respuesta basada en los resultados del análisis]

\textbf{Pregunta 2:} [Repetir la segunda pregunta de investigación]

\textbf{Respuesta:} [Proporcionar segunda respuesta basada en los resultados]

\textbf{Pregunta 3:} [Si hay más preguntas, agregarlas aquí]

\textbf{Respuesta:} [Proporcionar respuesta correspondiente]

\subsection{Cumplimiento de Objetivos}

[Evaluar el cumplimiento de cada objetivo específico planteado]

\subsection{Recomendaciones}

Basándose en los hallazgos del estudio, se proponen las siguientes recomendaciones:

\begin{enumerate}
    \item [Primera recomendación para la práctica o investigación futura]
    \item [Segunda recomendación relacionada con la metodología o el diseño]
    \item [Tercera recomendación para el desarrollo teórico del área]
    \item [Agregar más recomendaciones según sea necesario]
\end{enumerate}

\newpage

% ============================================
% REFERENCIAS - APA v7
% ============================================
\section{Referencias}

\printbibliography[heading=none]

% NOTA: Para citar en el texto, usa:
% \textcite{clave} para citas narrativas: "Autor (2020) afirma que..."
% \parencite{clave} para citas parentéticas: "... (Autor, 2020)"

\end{document}