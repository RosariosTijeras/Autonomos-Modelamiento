% ============================================
% EJEMPLOS DE CITACIÓN APA v7
% ============================================
% Copia y pega estos ejemplos en tu documento principal

% ============================================
% 1. CITAS NARRATIVAS (autor es parte de la oración)
% ============================================

% Un autor:
\textcite{wickham2016} desarrolló el paquete ggplot2 para visualización de datos.
% Resultado: Wickham (2016) desarrolló el paquete ggplot2...

% Dos autores:
\textcite{henderson1981} propusieron un método interactivo.
% Resultado: Henderson y Velleman (1981) propusieron...

% Tres o más autores:
\textcite{james2013} presentan técnicas de aprendizaje estadístico.
% Resultado: James et al. (2013) presentan...

% ============================================
% 2. CITAS PARENTÉTICAS (autor entre paréntesis)
% ============================================

% Una fuente:
La regresión lineal es una técnica fundamental \parencite{montgomery2012}.
% Resultado: ...técnica fundamental (Montgomery et al., 2012).

% Múltiples fuentes (se ordenan alfabéticamente):
Los métodos estadísticos \parencite{james2013, montgomery2012, kutner2005}.
% Resultado: ...métodos estadísticos (James et al., 2013; Kutner et al., 2005; Montgomery et al., 2012).

% ============================================
% 3. CITAS CON NÚMERO DE PÁGINA
% ============================================

% Cita narrativa con página:
\textcite[p.~45]{montgomery2012} explica que...
% Resultado: Montgomery et al. (2012, p. 45) explica que...

% Cita parentética con página:
"La correlación mide la fuerza de asociación" \parencite[p.~123]{cohen2003}.
% Resultado: ...asociación" (Cohen et al., 2003, p. 123).

% Rango de páginas:
\parencite[pp.~25-30]{james2013}
% Resultado: (James et al., 2013, pp. 25-30)

% ============================================
% 4. CITAS DE CITAS SECUNDARIAS
% ============================================

% Cuando citas un trabajo citado en otro trabajo:
\textcite{pearson1896}, citado en \textcite{cohen2003}, estableció...
% Resultado: Pearson (1896), citado en Cohen et al. (2003), estableció...

% ============================================
% 5. CITAS DE SOFTWARE Y PAQUETES
% ============================================

% Software R:
Todos los análisis se realizaron en R \parencite{rcore2023}.
% Resultado: ...en R (R Core Team, 2023).

% Paquete específico:
Se utilizó ggplot2 \parencite{wickham2016} para visualización.
% Resultado: ...ggplot2 (Wickham, 2016) para...

% ============================================
% 6. CITAS DE ARTÍCULOS
% ============================================

% Artículo de revista:
La prueba de Shapiro-Wilk \parencite{shapiro1965} evalúa normalidad.
% Resultado: ...Shapiro-Wilk (Shapiro & Wilk, 1965) evalúa...

% Artículo con DOI:
El test de Breusch-Pagan \parencite{breusch1979} detecta heterocedasticidad.
% Resultado: ...Breusch-Pagan (Breusch & Pagan, 1979) detecta...

% ============================================
% 7. CITAS EN TABLAS Y FIGURAS
% ============================================

\begin{table}[H]
    \centering
    \caption{Estadísticas Descriptivas de Variables \parencite{motor1974}}
    \begin{tabular}{lcc}
        \toprule \textbf{Variable} & \textbf{M} & \textbf{DE} \\
        \midrule Peso              & 3.22       & 0.98        \\
        \bottomrule
    \end{tabular}
    \label{tab:ejemplo}
\end{table}

% ============================================
% 8. MÚLTIPLES CITAS DEL MISMO AUTOR
% ============================================

% Si tienes varios trabajos del mismo autor en el mismo año,
% biblatex agregará automáticamente letras (2020a, 2020b)

% ============================================
% 9. CITAS EN NOTAS AL PIE (si las usas)
% ============================================

Este concepto es fundamental\footnote{\textcite{montgomery2012} proporciona una explicación
detallada.}.

% ============================================
% 10. CITAS SIN AUTOR (organizaciones)
% ============================================

% En el .bib usa:
% author = {{R Core Team}}  % Dobles llaves para mantener como organización

El lenguaje R \parencite{rcore2023} es ampliamente utilizado.
% Resultado: ...R (R Core Team, 2023) es...

% ============================================
% CONSEJOS IMPORTANTES:
% ============================================

% 1. SIEMPRE usa \textcite{} o \parencite{}, NO uses \cite{}
% 2. Compila 2-3 veces para que las citas se actualicen
% 3. Las referencias se generan automáticamente al final
% 4. Todas las citas deben estar en bibliografia.bib
% 5. APA usa "y" (and) entre autores, biblatex lo hace automáticamente
% 6. Para "et al." después de 3 autores, es automático en APA v7

% ============================================
% EJEMPLOS COMPLETOS EN CONTEXTO:
% ============================================

% Introducción con citas:
La regresión lineal simple es una técnica estadística fundamental \parencite{montgomery2012}
que permite modelar la relación entre dos variables. \textcite{james2013} describen
esta técnica como una de las más utilizadas en análisis de datos. El método fue
originalmente desarrollado por \textcite{pearson1896} y ha sido ampliamente estudiado
\parencite{kutner2005, cohen2003}.

% Metodología con citas:
Se utilizó el software R versión 4.3.0 \parencite{rcore2023} para realizar todos
los análisis estadísticos. La visualización de datos se realizó mediante el paquete
ggplot2 \parencite{wickham2016}, siguiendo los principios de gramática de gráficos.

% Resultados con citas:
Los datos provinieron del dataset mtcars \parencite{henderson1981, motor1974}, que
contiene información sobre 32 automóviles. La normalidad de los datos se evaluó
mediante la prueba de Shapiro-Wilk \parencite{shapiro1965}, mientras que la
homocedasticidad se verificó con el test de Breusch-Pagan \parencite{breusch1979}.

% Discusión con citas:
Estos resultados coinciden con los reportados por \textcite{montgomery2012} en
su análisis de modelos de regresión. Como señalan \textcite[p.~156]{james2013}, "la
correlación negativa entre peso y rendimiento es consistente con las leyes
físicas del movimiento".